%%%%%%%%%%%%%%%%%%%%%%%%%%%%%%%%%%%%%%%%%
%
% CMPT 435
% Fall 2021
% Lab One
%
%%%%%%%%%%%%%%%%%%%%%%%%%%%%%%%%%%%%%%%%%

%%%%%%%%%%%%%%%%%%%%%%%%%%%%%%%%%%%%%%%%%
% Short Sectioned Assignment
% LaTeX Template
% Version 1.0 (5/5/12)
%
% This template has been downloaded from: http://www.LaTeXTemplates.com
% Original author: % Frits Wenneker (http://www.howtotex.com)
% License: CC BY-NC-SA 3.0 (http://creativecommons.org/licenses/by-nc-sa/3.0/)
% Modified by Alan G. Labouseur  - alan@labouseur.com
%
%%%%%%%%%%%%%%%%%%%%%%%%%%%%%%%%%%%%%%%%%

%----------------------------------------------------------------------------------------
%	PACKAGES AND OTHER DOCUMENT CONFIGURATIONS
%----------------------------------------------------------------------------------------

\documentclass[letterpaper, 10pt,DIV=13]{scrartcl} 

\usepackage[T1]{fontenc} % Use 8-bit encoding that has 256 glyph
\usepackage[english]{babel} % English language/hyphenation
\usepackage{amsmath,amsfonts,amsthm,xfrac} % Math packages
\usepackage{sectsty} % Allows customizing section commands
\usepackage{graphicx}
\usepackage[lined,linesnumbered,commentsnumbered]{algorithm2e}
\usepackage{listings}
\usepackage{parskip}
\usepackage{lastpage}
\usepackage{multirow}

\allsectionsfont{\normalfont\scshape} % Make all section titles in default font and small caps.

\usepackage{fancyhdr} % Custom headers and footers
\pagestyle{fancyplain} % Makes all pages in the document conform to the custom headers and footers

\fancyhead{} % No page header - if you want one, create it in the same way as the footers below
\fancyfoot[L]{} % Empty left footer
\fancyfoot[C]{} % Empty center footer
\fancyfoot[R]{page \thepage\ of \pageref{LastPage}} % Page numbering for right footer

\renewcommand{\headrulewidth}{0pt} % Remove header underlines
\renewcommand{\footrulewidth}{0pt} % Remove footer underlines
\setlength{\headheight}{13.6pt} % Customize the height of the header

\numberwithin{equation}{section} % Number equations within sections (i.e. 1.1, 1.2, 2.1, 2.2 instead of 1, 2, 3, 4)
\numberwithin{figure}{section} % Number figures within sections (i.e. 1.1, 1.2, 2.1, 2.2 instead of 1, 2, 3, 4)
\numberwithin{table}{section} % Number tables within sections (i.e. 1.1, 1.2, 2.1, 2.2 instead of 1, 2, 3, 4)

\setlength\parindent{0pt} % Removes all indentation from paragraphs.

\binoppenalty=3000
\relpenalty=3000

%----------------------------------------------------------------------------------------
%	TITLE SECTION
%----------------------------------------------------------------------------------------

\newcommand{\horrule}[1]{\rule{\linewidth}{#1}} % Create horizontal rule command with 1 argument of height

\title{	
   \normalfont \normalsize 
   \textsc{CMPT 435 - Fall 2021 - Dr. Labouseur} \\[10pt] % Header stuff.
   \horrule{0.5pt} \\[0.25cm] 	% Top horizontal rule
   \huge Semester Project  \\     	    % Assignment title
   \horrule{0.5pt} \\[0.25cm] 	% Bottom horizontal rule
}

\author{Shannon Cordoni \\ \normalsize Shannon.Cordoni@Marist.edu}

\date{\normalsize\today} 	% Today's date.

\begin{document}
\maketitle % Print the title

%----------------------------------------------------------------------------------------
%   start PROBLEM ONE
%----------------------------------------------------------------------------------------
\section{Problem One: Semester Project}

\subsection{The Data Structure}
\paragraph{} Given Professor Labouseur's Covid Simulation Articles, our job was to duplicate his Covid testing protocol in our own simulation. \\


\subsection{Main Class}

\subsubsection{Description}
\paragraph{}This class is where most of our work is done, it takes in a parameter to be our population size and from there we infect the population at a 2\% infection rate and simulate the testing protocol to find out how many of each case types their are in our population. The methods contained in this class preform all these operations and more. 

\lstset{numbers=left, numberstyle=\tiny, stepnumber=1, numbersep=5pt, basicstyle=\footnotesize\ttfamily}
\begin{lstlisting}[frame=single, ] 

/*
 * 
 * Semester Project
 * Due Date and Time: 12/15/21 before 12:00am 
 * Purpose: to develop a simulation testing program for covid screenings
 * Input: The user will be inputting a population number to test the simulation on
 * Output: The program will output the infection rate of the population
 * @author Shannon Cordoni 
 * 
 */

import java.io.File;
import java.io.FileNotFoundException;
import java.util.*;


public class SimulationCordoni {

    //Declare keyboard 
    static Scanner keyboard = new Scanner(System.in);
    
    public static void main(String[] args) {

        
        //output population
        System.out.println("Population Simulation Number: " + args[0]);



        //Declare and initialize variables 
       
        //start with population being 1000
        int[] population = new int[Integer.parseInt(args[0])];

        //set the population size
        double populationSize = Double.parseDouble(args[0]);

        int groupNumber = 0;

        int groupSize = 8;
        
        double infectionRate = 0.02;

        double numberInfected = 0.0;

        int index = 0;

        int numberOfTests = 0;

        int numberOf1Tests = 0;

        int numberOf2Tests = 0;

        int numberOf3Tests = 0;


        int case1occurences = 0;

        int case2occurences = 0;

        int case3occurences = 0;

        int popindex = 0;

        //we are going to let 1 be infected and let 0 be healthy
        //This method is going to randomly infect the population

        numberInfected = populationSize * infectionRate;

        int intNumberInfected = (int) numberInfected;
        
        System.out.println("Number Infected: " + intNumberInfected);

        for (int i = 0; i < intNumberInfected ; i++){

            popindex = getRandomInfection(population);

            //if the index is already set to 1 we do not want to reinfect
            if(population[popindex] == 1){
                //System.out.println("Double Infection");
                
                //we keep calling random then until we find a non-infected person
                while(population[popindex] != 0){
                    //System.out.println("new Infection");
                    popindex = getRandomInfection(population);
                }//while

                population[popindex] = 1;

            }//if

            //otherwise we just infect
            else{
                population[popindex] = 1;
            }//else
            

        }//for
        

        //print out the population
        for (int i = 0; i < population.length; i++){

           //System.out.print(population[i]);
            
        }//for

        //split the population into groups of 8
        //want to make a 2 dimensional array so that we can go through each group of 8
        groupNumber = (Integer.parseInt(args[0])) / groupSize ;

        System.out.println("Group Number: " + groupNumber);
    
        //here we are going to make a copy of our group array to search through so that 
        //we still have an original 
        int[][] grouparray = new int [groupNumber][groupSize];

        int[][] groupArrayCopy = grouparray.clone();

       
        //go through the group array and input the population
        //then in each index of the group make an array the size of group size
        for (int i = 0; i < groupNumber; i++){

            for ( int j = 0; j < groupSize; j++){

               grouparray[i][j] = population[index];
               index++;
            }//for j
        }//for i

        //print out the group array to see each group of 8
        //System.out.println(Arrays.deepToString(grouparray));
                

        
        //now we test for infections!
        //testing 125 groups of 8
        for (int i = 0; i < groupNumber; i++){
            
            //go through and find each group sum to see if we need to individually test

            //find sum
            //System.out.println(findArraySum(groupArrayCopy[i]));

            //if the sum is equal to 0 then we dont have an infection
            if(findArraySum(groupArrayCopy[i]) == 0){
                    
            //System.out.println("No infection in this group!");
            numberOf1Tests++;
            case1occurences++;
                
                
            }//if  

            //else we have an infection
            else{

                int subgroup1infection = 0;
                int subgroup2infection = 0 ;

                    //now we split into 2 groups of 4 and retest
                    //Here we test the first 4 in the group
                    for(int k = 0; k < groupSize/2; k++){

                        numberOf2Tests++;

                        //if we find an infection then we set the variable to one
                        //so that we can see later if we have a case 3 infection
                        if(groupArrayCopy[i][k] == 1){
                           
                            subgroup1infection = 1;

                        }//if
    
                    }//for

                    //here we test the second 4 in the group
                    for(int k = 4; k < groupSize; k++){

                        numberOf2Tests++;
                        
                        //if we find an infection then we set the variable to one
                        //so that we can see later if we have a case 3 infection
                        if(groupArrayCopy[i][k] == 1){
                            
                            subgroup2infection = 1;

                        }//if
    
                    }//for

                    //If there is both a subgroup 1 and subgroup 2 infection then we 
                    //have a case 3
                    if((subgroup1infection == 1) && (subgroup2infection == 1)){

                        case3occurences++;
                        
                        //here we account for the original test of all 8, and the two 
                        //tests for each group of 4
                        numberOf3Tests = numberOf3Tests + 3;

                        //we loop through for testing all 8 in the group
                        for(int j = 0; j < groupSize; j++){

                            numberOf3Tests++;
                            
                        }//for

                    }//if both subgroup


                    //if we only have a subgroup 1 or subgroup 2 infection then we
                    //have a case 2
                    else if(subgroup1infection == 1){

                        case2occurences++;

                    }//if subgroup 1

                    else if(subgroup2infection == 1){

                        case2occurences++;

                    }//if subgroup 2

            }//else

        }//for i 


        //print out the results
        System.out.println("Case 1 occurences: " +  case1occurences);
        System.out.println("Case 2 occurences: " +  case2occurences);
        System.out.println("Case 3 occurences: " +  case3occurences);
        
        System.out.println("Number of case 1 tests: " +  numberOf1Tests);
        System.out.println("Number of case 2 tests: " +  numberOf2Tests);
        System.out.println("Number of case 3 tests: " +  numberOf3Tests);

        numberOfTests = numberOf1Tests + numberOf2Tests + numberOf3Tests;

        System.out.println("Number of total tests: " +  numberOfTests);
        
    }//main

    //This method randomly infects the population
    public static int getRandomInfection(int[] populationArray)
    {
        

        int randomIndex = 0;

        Random randomize = new Random();

        randomIndex = randomize.nextInt(populationArray.length);

        //System.out.println("random index" + randomIndex);

        return randomIndex;
        
    }//getRandomInfection

    //this method find the sum of the array passed in
    public static int findArraySum(int[] array) {
        
        int sum = 0;

        for (int i = 0; i < array.length; i++) {
            sum = sum + array[i];
        }//for

        return sum;
    }//find array sum

}//semester cordoni
 
\end{lstlisting}

\subsubsection{Description of Main Code}
\paragraph{} The main class above consists of different methods to help simulate the Covid testing protocol. The good parts of the code first include taking in the population parameter and setting it to be the population size. Then we make an array the size of our population and "infect" it, calling our $getRandomInfection$ method to return random indexes to set as "$1$" to signify an infection. At each pass we check to see if our index is already infected before we possibly re-infect someone. We do this for $2\%$ of whatever population was passed in as a parameter. Then we split this into groups of 8 and we create a 2 dimensional array with the first dimension being the group number and the second dimension being the group size.  Along with passing each line of this 2 dimensional group array into our testing simulation to see how many occurences of each case there was and how many tests it took to test for each case. Case 1 being that there were 0 infections in the group, case 2 being that there was 1 infection in the group, and case 3 being that there was more than one infection in the group. Then to keep the non-simulation code out of the main method, different methods were used to help organize the code better. These methods include the $getRandomInfection$, and $findArraySum$ method.

\paragraph{} To go into more detail of the actual simulation testing we go through each group of 8 and we take the sum of each group. If the group sum is 0 then we have a case 1 occurrence and 0 infections present. If the group sum is greater than 0 then we have an infection present and we must do further testing to see how many infections are present. First we loop through the first group of 4 and see if there are any infections present, during this time we increase the numbers of case 2 tests and if an infection is found then we set our $subgroup1infection$ variable to 1 for later. Then we do the same for the second group of 4 and if an infection is found then we set our $subgroup2infection$ variable to 1 for later. If both variables are set to 1 then we have a case 3 occurrence, of which there is more than 1 infection in the group. If only one of the variables is set to 1 then we have a case 2 occurrence. After testing we print out our results, and keep increasing the population size to see if their are any trends. 

\paragraph{} The $getRandomInfection$ method takes in an array representing the population, and gets a random index in this array to return to the main method for infection. The index returned then turns from 0 to 1 to represent an infection.

\paragraph{} The $findArraySum$ method takes in an array, and returns the sum of it. This comes into play when initially determining if there is an infection in each group. If the array sum is 0 then there is no infection, however, if it is 1 then we need to do further testing to determine if the group falls under a case 2 occurrence or a case 3 occurrence. 

\newpage

\subsection{Overall:}

\paragraph{} Overall, the simulation testing representation was successful in implementation. To go through each population possibility we can create a table for better data understanding, this table will show some population possibilities:

\begin{center}
\begin{tabular}{ |p{4.5cm}||p{4.5cm}||p{4.5cm}|  }
 \hline
 \multicolumn{3}{|c|}{Populations} \\
 \hline
 1000 & 10000 & 100000\\
 \hline
 Population: 1000  & Population: 10000 &  Population: 100000\\
 Number Infected: 20 & Number Infected: 200 & Number Infected: 2000\\
 Group Number: 125 & Group Number: 1250 & Group Number: 12500\\
 Case 1 occurences: 106 & Case 1 occurences: 1067 & Case 1 occurences: 10655\\
 Case 2 occurences: 18 & Case 2 occurences: 177 & Case 2 occurences: 1767 \\
 Case 3 occurences: 1 & Case 3 occurences: 6 & Case 3 occurences: 78 \\
 Number of case 1 tests: 106 & Number of case 1 tests: 1067 & Number of case 1 tests: 10655 \\
 Number of case 2 tests: 152 & Number of case 2 tests: 1464 & Number of case 2 tests: 14760 \\
 Number of case 3 tests: 11 & Number of case 3 tests: 66 & Number of case 3 tests: 858 \\
 Number of total tests: 269 & Number of total tests: 2597 & Number of total tests: 26273 \\
 
 \hline
\end{tabular}
\end{center}

\paragraph{} The table above shows a quick understanding of the population possibilities. As the results get bigger for the increasing populations the results follow more of a hyper geometric distribution vs a binomial distribution. The difference between the two distributions is that the hyper geometric distribution is sampling without replacement, meaning the probabilities change at each turn and depend on the result of the previous turn. The binomial distribution is sampling with replacement meaning that each turn is independent of each other and have an equal chance of happening. For the future there are always many ways to improve ones code, for this simulation I am sure there is a better way to go about keeping track of how many tests are involved with each case occurrence, along with overall simplification of the code to be more effective. Overall, this simulation assignment went smoothly and was successful in implementation.

\end{document}
